\HeaderA{read.dbf}{Read the Attribute (dbf) File of an ESRI Shapefile}{read.dbf}
\keyword{survey}{read.dbf}
\begin{Description}\relax
This function reads either a single dbf file or multiple dbf files.  For 
multiple dbf files, all of the dbf files must have the same variable names.
\end{Description}
\begin{Usage}
\begin{verbatim}
read.dbf(filename=NULL)
\end{verbatim}
\end{Usage}
\begin{Arguments}
\begin{ldescription}
\item[\code{filename}] name of the dbf file without any extension.  If filename 
equals a dbf file name, then that dbf file is read.  If filename 
equals NULL, then all of the dbf files in the current directory are 
read.  The default is NULL.
\end{ldescription}
\end{Arguments}
\begin{Details}\relax
Function summary(), i.e., summary.SurveyFrame(), can be used to summarize the 
the frame for a survey design.
\end{Details}
\begin{Value}
A data frame composed of either the contents of the single dbf file, when
filename is provided, or the contents of the dbf file(s) in the current
directory, when filename is NULL.  The data frame is assigned class
"SurveyFrame".
\end{Value}
\begin{Author}\relax
Tom Kincaid \email{Kincaid.Tom@epa.gov}
\end{Author}
\begin{References}\relax
ESRI Shapefile Technical Description: 
\url{http://www.esri.com/library/whitepapers/pdfs/shapefile.pdf}
\end{References}
\begin{SeeAlso}\relax
\code{\LinkA{read.shape}{read.shape}}
\code{\LinkA{framesum}{framesum}}
\end{SeeAlso}
\begin{Examples}
\begin{ExampleCode}
  ## Not run: 
  read.shape("my.dbffile")
  
## End(Not run)
\end{ExampleCode}
\end{Examples}

