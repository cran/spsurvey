\HeaderA{read.shape}{Read an ESRI Shapefile}{read.shape}
\keyword{IO}{read.shape}
\begin{Description}\relax
This function reads either a single shapefile or multiple shapefiles.  For 
multiple shapefiles, all of the shapefiles must be the same type, i.e., 
point, polyline, or polygon.
\end{Description}
\begin{Usage}
\begin{verbatim}
read.shape(filename=NULL)
\end{verbatim}
\end{Usage}
\begin{Arguments}
\begin{ldescription}
\item[\code{filename}] name of the shapefile without any extension.  If filename 
equals a shapefile name, than that shapefile is read.  If filename 
equals NULL, then all of the shapefiles in the current directory are 
read.  The default is NULL.
\end{ldescription}
\end{Arguments}
\begin{Value}
An sp package object containing information in the shapefile.  The object is
assigned class "SpatialPointsDataFrame", "SpatialLinesDataFrame", or
"SpatialPolygonsDataFrame" corresponding to the shapefile type, i.e., point,
polyline, or polygon, respectively.  For further information regarding the
output object, see documentation for the sp package.
\end{Value}
\begin{Author}\relax
Tom Kincaid \email{Kincaid.Tom@epa.gov}
\end{Author}
\begin{References}\relax
ESRI Shapefile Technical Description: 
\url{http://www.esri.com/library/whitepapers/pdfs/shapefile.pdf}
\end{References}
\begin{SeeAlso}\relax
\code{\LinkA{read.dbf}{read.dbf}}
\end{SeeAlso}
\begin{Examples}
\begin{ExampleCode}
  ## Not run: 
  read.shape("my.shapefile")
  
## End(Not run)
\end{ExampleCode}
\end{Examples}

