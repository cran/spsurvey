\HeaderA{category.est}{Category Proportion and Size Estimates}{category.est}
\keyword{survey}{category.est}
\keyword{univar}{category.est}
\begin{Description}\relax
This function estimates proportion (expressed as percent) and size of a
resource in each of a set of categories and can also be used to estimate
proportion and size for site status categories.  Upper and lower confidence
bounds also are estimated.
\end{Description}
\begin{Usage}
\begin{verbatim}
category.est(catvar, wgt, x=NULL, y=NULL, stratum=NULL, cluster=NULL,
   N.cluster=NULL, wgt1=NULL, x1=NULL, y1=NULL, popsize=NULL, stage1size=NULL,
   support=NULL, swgt=NULL, swgt1=NULL, unitsize=NULL, vartype="Local", conf=95,
   check.ind=TRUE, warn.ind=NULL, warn.df=NULL, warn.vec=NULL)
\end{verbatim}
\end{Usage}
\begin{Arguments}
\begin{ldescription}
\item[\code{catvar}] the value of the categorical response variable or the site
status for each site.
\item[\code{wgt}] the final adjusted weight (inverse of the sample inclusion
probability) for each site, which is either the weight for a 
single-stage sample or the stage two weight for a two-stage sample.
\item[\code{x}] x-coordinate for location for each site, which is either the
x-coordinate for a single-stage sample or the stage two 
x-coordinate for a two-stage sample.  The default is NULL.
\item[\code{y}] y-coordinate for location for each site, which is either the
y-coordinate for a single-stage sample or the stage two 
y-coordinate for a two-stage sample.  The default is NULL.
\item[\code{stratum}] the stratum for each site.  The default is NULL.
\item[\code{cluster}] the stage one sampling unit (primary sampling unit or cluster) 
code for each site.  The default is NULL.
\item[\code{N.cluster}] the number of stage one sampling units in the resource, which 
is required for calculation of finite and continuous population 
correction factors for a two-stage sample.  For a stratified sample 
this variable must be a vector containing a value for each stratum and
must have the names attribute set to identify the stratum codes.  The
default is NULL.
\item[\code{wgt1}] the final adjusted stage one weight for each site.  The default
is NULL.
\item[\code{x1}] the stage one x-coordinate for location for each site.  The default
is NULL.
\item[\code{y1}] the stage one y-coordinate for location for each site.  The default
is NULL.
\item[\code{popsize}] the known size of the resource - the total number of sampling 
units of a finite resource or the measure of an extensive resource,
which is required for calculation of finite and continuous population 
correction factors for a single-stage sample.  This variable is also 
used to adjust estimators for the known size of a resource.  For a
stratified sample this variable must be a vector containing a value 
for each stratum and must have the names attribute set to identify the
stratum codes.  The default is NULL.
\item[\code{stage1size}] the known size of the stage one sampling units of a two-
stage sample, which is required for calculation of finite and  
continuous population correction factors for a two-stage sample and 
must have the names attribute set to identify the stage one sampling 
unit codes.  For a stratified sample, the names attribute must be set
to identify both stratum codes and stage one sampling unit codes using
a convention where the two codes are separated by the \# symbol, e.g.,
"Stratum 1\#Cluster 1".  The default is NULL.
\item[\code{support}] the support value for each site - the value one (1) for a 
site from a finite resource or the measure of the sampling unit  
associated with a site from an extensive resource, which is required  
for calculation of finite and continuous population correction  
factors.  The default is NULL.
\item[\code{swgt}] the size-weight for each site, which is the stage two size-weight 
for a two-stage sample.  The default is NULL.
\item[\code{swgt1}] the stage one size-weight for each site.  The default is NULL.
\item[\code{unitsize}] the known sum of the size-weights of the resource, which for a 
stratified sample must be a vector containing a value for each stratum 
and must have the names attribute set to identify the stratum codes.  
The default is NULL.
\item[\code{vartype}] the choice of variance estimator, where "Local" = local mean
estimator and "SRS" = SRS estimator.  The default is "Local".
\item[\code{conf}] the confidence level.  The default is 95\%.
\item[\code{check.ind}] a logical value that indicates whether compatability
checking of the input values is conducted, where TRUE = conduct 
compatibility checking and FALSE = do not conduct compatibility 
checking.  The default is TRUE.
\item[\code{warn.ind}] a logical value that indicates whether warning messages were
generated, where TRUE = warning messages were generated and FALSE = warning
messages were not generated.  The default is NULL.
\item[\code{warn.df}] a data frame for storing warning messages.  The default is
NULL.
\item[\code{warn.vec}] a vector that contains names of the population type, the
subpopulation, and an indicator.  The default is NULL.
\end{ldescription}
\end{Arguments}
\begin{Details}\relax
Proportion estimates are calculated using the Horvitz-Thompson ratio estimator,
i.e., the ratio of two Horvitz-Thompson estimators.  The numerator of the ratio
estimates the size of the category.  The denominator of the ratio
estimates the size of the resource.   Variance estimates for the proportion
estimates are calculated using either the local mean variance estimator or the
simple random sampling (SRS) variance estimator.  The choice of variance
estimator is subject to user control.  The local mean variance estimator
requires the x-coordinate and the y-coordinate of each site.  The
SRS variance estimator uses the independent  random sample approximation
to calculate joint inclusion probabilities.   Confidence bounds are calculated
using a Normal distribution multiplier.  For a finite resource size is
the number of units in the resource.  For an extensive resource size is the 
measure (extent) of the resource, i.e., length, area, or volume.  Size 
estimates are calculated using the Horvitz- Thompson estimator.  Variance 
estimates for the size estimates are calculated using either the local mean 
variance estimator or the SRS variance estimator. The function can 
accommodate a stratified sample.  For a stratified sample, separate estimates 
and standard errors are calculated for each stratum, which are used to produce 
estimates and standard errors for all strata combined.  Strata that contain a 
single value are removed.  For a stratified sample, when either the size of the
resource or the sum of the size-weights for the resource is provided for each 
stratum, those values are used as  stratum weights for calculating the 
estimates and standard errors for all strata combined.  In addition, when 
either of those known values is provided for each stratum, size estimates are 
obtained by multiplying the proportion estimate, i.e., the Horvitz-Thompson 
ratio estimator, by the known value for the stratum.  For a stratified sample 
when neither the size of the  resource nor the sum of the size-weights of the 
resource is provided for  each stratum, estimated values are used as stratum 
weights for calculating  the estimates and standard errors for all strata 
combined.  The function can accommodate single-stage and two-stage samples for 
both stratified and unstratified sampling designs.  Finite population and 
continuous population correction factors can be utilized in variance 
estimation.  The function checks for compatibility of input values and removes
missing values.
\end{Details}
\begin{Section}{Value}
If the function was called by the cat.analysis function, then value is a list
containing the following components:
\Itemize{
\item \code{Results} - a data frame containing estimates and confidence
bounds
\item \code{warn.ind} - a logical value indicating whether warning messages
were generated
\item \code{warn.df} - a data frame containing warning messages
}
If the function was called directly, then value is a data frame containing
estimates and confidence bounds.
\end{Section}
\begin{Author}\relax
Tom Kincaid \email{Kincaid.Tom@epa.gov}
\end{Author}
\begin{References}\relax
Diaz-Ramos, S., D.L. Stevens, Jr., and A.R. Olsen. (1996).  \emph{EMAP
Statistical Methods Manual.} EPA/620/R-96/XXX.  Corvallis, OR: U.S.
Environmental Protection Agency, Office of Research and Development, National
Health Effects and Environmental Research Laboratory, Western Ecology
Division.
\end{References}
\begin{Examples}
\begin{ExampleCode}
catvar <- rep(c("north", "south", "east", "west"), rep(25, 4))
wgt <- runif(100, 10, 100)
category.est(catvar, wgt, vartype="SRS")

x <- runif(100)
y <- runif(100)
category.est(catvar, wgt, x, y)
\end{ExampleCode}
\end{Examples}

