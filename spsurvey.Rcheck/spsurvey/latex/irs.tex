\HeaderA{irs}{Independent Random Sample (IRS) Survey Design}{irs}
\keyword{survey}{irs}
\begin{Description}\relax
Selects an independent random sample (IRS) survey design. The IRS survey
design may include stratification, unequal probability using categories,
unequal selection proportional to an auxiliary variable, survey over time
structures, and provision for an oversample.
\end{Description}
\begin{Usage}
\begin{verbatim}
irs(design, DesignID="Site", SiteBegin=1, type.frame="finite",
   src.frame="shapefile", in.shape=NULL, sp.object=NULL, att.frame=NULL,
   id=NULL, xcoord=NULL, ycoord=NULL, stratum=NULL, mdcaty=NULL, maxtry=1000,
   shapefile=TRUE, prjfilename=NULL, out.shape="sample")
\end{verbatim}
\end{Usage}
\begin{Arguments}
\begin{ldescription}
\item[\code{design}] named list of stratum design specifications, where each element of
design is a list containing the design specifications for a stratum.  For
an unstratified sample, design contains a single list.  If the sample is
stratified, the names in design must occur among the strata names in the
stratum column of the attributes data frame (att.frame).  If the sample is
unstratified, the name of the single list in design is arbitrary.  Each
list in design has four components:\\
panel = named vector of sample sizes for each panel in stratum\\
seltype = the type of random selection, which must be one of following:
"Equal" - equal probability selection, "Unequal" - unequal probability
selection by the categories specified in caty.n and mdcaty, or
"Continuous" - unequal probability selection proportional to auxiliary
variable mdcaty\\
caty.n = if seltype equals "Unequal", a named vector of sample sizes for
each category specified by mdcaty, where sum of the sample sizes must
equal sum of the panel sample sizes, and names must be a subset of
values in mdcaty\\
over = number of replacement sites ("oversample" sites) for the entire
design, which is set equal to 0 if none are required\\\\
Example design for a stratified sample:\\
design = list("Stratum 1"=list(panel=c(Panel=50), seltype="Equal",
over=10),\\ "Stratum 2"=list(panel=c("Panel One"=50, "Panel Two"=50),
seltype="Unequal",\\ caty.n=c(CatyOne=25, CatyTwo=25, CatyThree=25,
CatyFour=25), over=75))\\\\
Example design for an unstratified sample:\\
design = list(None=list(panel=c(Panel1=50, Panel2=100, Panel3=50),
seltype="Unequal",\\ caty.n=c("Caty 1"=50, "Caty 2"=25, "Caty 3"=25,
"Caty 4"=25, "Caty 5"=75), over=100))\\
\item[\code{DesignID}] name for the design, which is used to create a site
ID for each site.  The default is "Site".
\item[\code{SiteBegin}] number to use for first site in the design.  The default is
1.
\item[\code{type.frame}] the type of frame, which must be one of following: "finite",
"linear", or "area".  The default is "finite".
\item[\code{src.frame}] source of the frame, which equals "shapefile" if the frame is
to be read from a shapefile, "sp.object" if the frame is obtained from an sp
package object, or "att.frame" if type.frame equals "finite" and the frame
is included in att.frame.  The default is "shapefile".
\item[\code{in.shape}] name (without any extension) of the input shapefile.  If
src.frame equal "shapefile" and in.shape equals NULL, then the shapefile or
shapefiles in the currrent directory are used.  The default is NULL.
\item[\code{sp.object}] name of the sp package object when src.frame equals
"sp.object".  The default is NULL.
\item[\code{att.frame}] a data frame composed of attributes associated with elements
in the frame, which must contain the columns used for stratum and mdcaty (if 
required).  If src.frame equals "shapefile" and att.frame equals NULL, then
att.frame is created from the dbf file(s) in the current directory.  If
src.frame equals "att.frame", then att.frame includes columns that contain
x-coordinates and y-coordinates for each element in the frame.  The default
is NULL.
\item[\code{id}] name of the column from att.frame that identifies the ID value for
each element in the frame.  If id equals NULL, a column named "id" that
contains values from one through the number of rows in att.frame is added to
att.frame.  The default is NULL.
\item[\code{xcoord}] name of the column from att.frame that identifies x-coordinates
when src.frame equals "att.frame".  If xcoord equals NULL, then xcoord is
given the value "x".  The default is NULL.
\item[\code{ycoord}] name of the column from att.frame that identifies y-coordinates
when src.frame equals "att.frame".  If ycoord equals NULL, then ycoord is
given the value "y".  The default is NULL.
\item[\code{stratum}] name of the column from att.frame that identifies stratum
membership for each element in the frame.  If stratum equals NULL, the
design is unstratified, and a column named "stratum" (with all its elements
equal to the stratum name specified in design) is added to att.frame.  The
default is NULL.
\item[\code{mdcaty}] name of the column from att.frame that identifies the unequal
probability category for each element in the frame.  The default is
NULL.
\item[\code{maxtry}] maximum number of iterations for randomly generating a point
within the frame to select a site when type.frame equals "area".  The
default is 1000.
\item[\code{shapefile}] option to create a shapefile containing the survey design
information, where TRUE equals create a shapefile and FALSE equals do not
create a shapefile.  The default is TRUE.
\item[\code{prjfilename}] name (without any extension) of the project file for an
input shapefile.  The default is NULL.
\item[\code{out.shape}] name (without any extension) of the output shapefile
containing the survey design information.  The default is "sample".
\end{ldescription}
\end{Arguments}
\begin{Details}\relax
The IRS survey design process selects a sample based on the survey design 
specification.\\\\
Function dsgnsum(), can be used to summarize the sites selected for a survey
design.
\end{Details}
\begin{Value}
An sp package object containing the survey design information and any
additional attribute variables that were provided.  The object is assigned
class "SpatialPointsDataFrame".  For further information regarding the
output object, see documentation for the sp package.  Optionally, a
shapefile can be created that contains the survey design information.
\end{Value}
\begin{Author}\relax
Tom Kincaid \email{Kincaid.Tom@epa.gov}
\end{Author}
\begin{SeeAlso}\relax
\code{\LinkA{irspts}{irspts}}
\code{\LinkA{irslin}{irslin}}
\code{\LinkA{irsarea}{irsarea}}
\code{\LinkA{dsgnsum}{dsgnsum}}
\end{SeeAlso}
\begin{Examples}
\begin{ExampleCode}
## Not run: 
# The following example will select a sample from an area resource.  The design
# includes two strata.  For Stratum 1, a sample of size 50 will be selected for
# a single panel.  For Stratum 2, a sample of size 50 will be selected for each
# of two panels.  In addition both strata will include oversamples (size 10 for
# Stratum 1 and size 75 for Stratum 2).  It is assumed that a shapefile defining
# the polygons for the area resource is located in the folder from which R is
# started.  Attribute data for the design will be read from the dbf file of the
# shapefile, which is assumed to have a variable named "test.stratum" that
# specifies stratum membership value for each record in the shapefile. A
# shapefile named "test.sample" containing the survey design information will be
# created.
test.design <- list("Stratum 1"=list(panel=c(Panel=50), seltype="Equal",
   over=10), "Stratum 2"=list(panel=c("Panel One"=50, "Panel Two"=50),
   seltype="Unequal", caty.n=c(CatyOne=25, CatyTwo=25, CatyThree=25,
   CatyFour=25), over=75))
test.attframe <- read.dbf("test.shapefile")
test.sample <- irs(design=test.design, DesignID="Test.Site", type.frame="area",
   src.frame="shapefile", in.shape="test.shapefile", att.frame=test.attframe,
   stratum="test.stratum", mdcaty="test.mdcaty", shapefile=TRUE,
   out.shape="test.sample")
## End(Not run)
\end{ExampleCode}
\end{Examples}

