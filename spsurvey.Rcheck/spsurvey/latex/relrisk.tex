\HeaderA{relrisk}{Relative Risk}{relrisk}
\keyword{survey}{relrisk}
\keyword{survival}{relrisk}
\begin{Description}\relax
This function calculates the relative risk estimate for a 2x2 table of cell
counts defined by a categorical response variable and a categorical
explanatory (stressor) variable for an unequal probability design.  Relative
risk is the ratio of two probabilities: the numerator is the probability that
the first level of the response variable is observed given occurrence of the
first level of the stressor variable, and the denominator is the probability
that the first level of the response variable is observed given occurrence of
the second level of the stressor variable.  The standard error of the base e
log of the relative risk estimate and confidence limits for the estimate also
are calculated.
\end{Description}
\begin{Usage}
\begin{verbatim}
relrisk(dframe, response="response", stressor="stressor",
   response.levels=c("Poor", "Good"), stressor.levels=c("Poor", "Good"),
   wgt="wgt", xcoord="xcoord", ycoord="ycoord", stratum=NULL, cluster=NULL,
   N.cluster=NULL, wgt1=NULL, xcoord1=NULL, ycoord1=NULL, popsize=NULL,
   stage1size=NULL, support=NULL, swgt=NULL, swgt1=NULL, unitsize=NULL,
   vartype="Local", conf=95, check.ind=TRUE)
\end{verbatim}
\end{Usage}
\begin{Arguments}
\begin{ldescription}
\item[\code{dframe}] a data frame containing the variables required for the analysis.
If variable names are not provided in the corresponding arguments,
then variables should be named as follows:\\
response = the categorical response variable values\\
stressor = the categorical explanatory (stressor) variable values\\
wgt = the final adjusted weights\\
xcoord = the x-coordinates for location\\
ycoord = the y-coordinates for location\\
stratum = the stratum codes
\item[\code{response}] name of the column in dframe containing the categorical
response variable.  The default is "response".
\item[\code{stressor}] name of the column in dframe containing the categorical
stressor variable.  The default is "stressor".
\item[\code{response.levels}] category values (levels) for the categorical response
variable, where the first level is used for calculating the relative risk 
estimate.  If response.levels is not supplied, then values "Poor" and "Good"
are used for the first level and second level of the response variable,
respectively.  The default is c("Poor", "Good").
\item[\code{stressor.levels}] category values (levels) for the categorical stressor
variable, where the first level is used for calculating the numerator of 
the relative risk estimate and the second level is used for calculating 
the denominator of the estimate.  If stressor.levels is not supplied, then
values "Poor" and "Good" are used for the first level and second level of
the stressor variable, respectively.  The default is c("Poor", "Good").
\item[\code{wgt}] the final adjusted weight (inverse of the sample inclusion
probability) for each site, which is either the weight for a 
single-stage sample or the stage two weight for a two-stage sample.
\item[\code{xcoord}] x-coordinate for location for each site, which is either the
x-coordinate for a single-stage sample or the stage two 
x-coordinate for a two-stage sample.  The default is NULL.
\item[\code{ycoord}] y-coordinate for location for each site, which is either the
y-coordinate for a single-stage sample or the stage two 
y-coordinate for a two-stage sample.  The default is NULL.
\item[\code{stratum}] the stratum for each site.  The default is NULL.
\item[\code{cluster}] the stage one sampling unit (primary sampling unit or cluster) 
code for each site.  The default is NULL.
\item[\code{N.cluster}] the number of stage one sampling units in the resource, which 
is required for calculation of finite and continuous population 
correction factors for a two-stage sample.  For a stratified sample 
this variable must be a vector containing a value for each stratum and
must have the names attribute set to identify the stratum codes.  The
default is NULL.
\item[\code{wgt1}] the final adjusted stage one weight for each site.  The default
is NULL.
\item[\code{xcoord1}] the stage one x-coordinate for location for each site.  The
default is NULL.
\item[\code{ycoord1}] the stage one y-coordinate for location for each site.  The
default is NULL.
\item[\code{popsize}] the known size of the resource - the total number of sampling 
units of a finite resource or the measure of an extensive resource,
which is required for calculation of finite and continuous population 
correction factors for a single-stage sample.  This variable is also 
used to adjust estimators for the known size of a resource.  For a
stratified sample this variable must be a vector containing a value 
for each stratum and must have the names attribute set to identify the
stratum codes.  The default is NULL.
\item[\code{stage1size}] the known size of the stage one sampling units of a two-
stage sample, which is required for calculation of finite and  
continuous population correction factors for a two-stage sample and 
must have the names attribute set to identify the stage one sampling 
unit codes.  For a stratified sample, the names attribute must be set
to identify both stratum codes and stage one sampling unit codes using
a convention where the two codes are separated by the \# symbol, e.g.,
"Stratum 1\#Cluster 1".  The default is NULL.
\item[\code{support}] the support value for each site - the value one (1) for a 
site from a finite resource or the measure of the sampling unit  
associated with a site from an extensive resource, which is required  
for calculation of finite and continuous population correction  
factors.  The default is NULL.
\item[\code{swgt}] the size-weight for each site, which is the stage two size-weight 
for a two-stage sample.  The default is NULL.
\item[\code{swgt1}] the stage one size-weight for each site.  The default is NULL.
\item[\code{unitsize}] the known sum of the size-weights of the resource, which for a 
stratified sample must be a vector containing a value for each stratum 
and must have the names attribute set to identify the stratum codes.  
The default is NULL.
\item[\code{vartype}] the choice of variance estimator, where "Local" = local mean
estimator and "SRS" = SRS estimator.  The default is "Local".
\item[\code{conf}] the confidence level.  The default is 95\%.
\item[\code{check.ind}] a logical value that indicates whether compatability
checking of the input values is conducted, where TRUE = conduct 
compatibility checking and FALSE = do not conduct compatibility 
checking.  The default is TRUE.
\end{ldescription}
\end{Arguments}
\begin{Details}\relax
The relative risk estimate is computed using the ratio of a numerator
probability to a denominator probability, which are estimated using cell and
marginal totals from a 2x2 table of cell counts defined by a categorical
response variable and a categorical stressor variable. An estimate of the
numerator probability is provided by the ratio of the cell total defined by
the first level of response variable and the first level of the stressor
variable to the marginal total for the first level of the stressor variable.
An estimate of the denominator probability is provided by the ratio of the
cell total defined by the first level of response variable and the second
level of the stressor variable to the marginal total for the second level of
the stressor variable.  Cell and marginal totals are estimated using the
Horvitz-Thompson estimator. The standard error of the base e log of the
relative risk estimate is calculated using a first-order Taylor series
linearization (Sarndal et al., 1992).
\end{Details}
\begin{Section}{Value}
Value is a list containing the following components:
\Itemize{
\item \code{RelRisk} - the relative risk estimate
\item \code{RRnum} - numerator ("elevated" risk) of the relative risk
estimate
\item \code{RRdenom} - denominator ("baseline" risk) of the relative risk
estimate
\item \code{RRlog.se} - standard error for the log of the relative risk
estimate
\item \code{ConfLimits} - confidence limits for the relative risk estimate
\item \code{WeightTotal} - sum of the final adjusted weights
\item \code{CellCounts} - cell and margin counts for the 2x2 table
\item \code{CellProportions} - estimated cell proportions for the 2x2 table
}
\end{Section}
\begin{Author}\relax
Tom Kincaid \email{Kincaid.Tom@epa.gov}
\end{Author}
\begin{References}\relax
S�rndal, C.-E., B. Swensson, and J. Wretman. (1992). \emph{Model Assisted
Survey Sampling.} Springer-Verlag, New York.
\end{References}
\begin{Examples}
\begin{ExampleCode}
dframe <- data.frame(response=sample(c("Poor", "Good"), 100, replace=TRUE),
   stressor=sample(c("Poor", "Good"), 100, replace=TRUE),
   wgt=runif(100, 10, 100))
relrisk(dframe, vartype="SRS")

dframe$xcoord <- runif(100)
dframe$ycoord <- runif(100)
relrisk(dframe)
\end{ExampleCode}
\end{Examples}

