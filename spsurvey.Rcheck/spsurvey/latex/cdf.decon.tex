\HeaderA{cdf.decon}{Cumulative Distribution Function - Deconvolution}{cdf.decon}
\keyword{survey}{cdf.decon}
\keyword{distribution}{cdf.decon}
\begin{Description}\relax
This function calculates an estimate of the deconvoluted cumulative 
distribution function (CDF) for the proportion (expressed as percent) and 
the total of a response variable, where the response variable may be defined
for either a finite or an extensive resource.  Optionally, for a finite 
resource, the size-weighted CDF can be calculated.  In addition the standard 
error of the estimated CDF and confidence bounds are calculated.  The 
simulation extrapolation deconvolution method (Stefanski and Bay, 1996) is
used to deconvolute measurement error variance from the response.
\end{Description}
\begin{Usage}
\begin{verbatim}
cdf.decon(z, wgt, sigma, var.sigma=NULL, x=NULL, y=NULL, stratum=NULL,
   cluster=NULL, N.cluster=NULL, wgt1=NULL, x1=NULL, y1=NULL, popsize=NULL,
   stage1size=NULL, support=NULL, swgt=NULL, swgt1=NULL, unitsize=NULL,
   vartype="Local", conf=95, cdfval=NULL, pctval=c(5,10,25,50,75,90,95),
   check.ind=TRUE, warn.ind=NULL, warn.df=NULL, warn.vec=NULL)
\end{verbatim}
\end{Usage}
\begin{Arguments}
\begin{ldescription}
\item[\code{z}] the response value for each site.
\item[\code{wgt}] the final adjusted weight (inverse of the sample inclusion
probability) for each site, which is either the weight for a 
single-stage sample or the stage two weight for a two-stage sample.
\item[\code{sigma}] measurement error variance.
\item[\code{var.sigma}] variance of the measurement error variance.  The default is
NULL.
\item[\code{x}] x-coordinate for location for each site, which is either the
x-coordinate for a single-stage sample or the stage two 
x-coordinate for a two-stage sample.  The default is NULL.
\item[\code{y}] y-coordinate for location for each site, which is either the
y-coordinate for a single-stage sample or the stage two 
y-coordinate for a two-stage sample.  The default is NULL.
\item[\code{stratum}] the stratum for each site.  The default is NULL.
\item[\code{cluster}] the stage one sampling unit (primary sampling unit or cluster) 
code for each site.  The default is NULL.
\item[\code{N.cluster}] the number of stage one sampling units in the resource, which 
is required for calculation of finite and continuous population 
correction factors for a two-stage sample.  For a stratified sample 
this variable must be a vector containing a value for each stratum and
must have the names attribute set to identify the stratum codes.  The
default is NULL.
\item[\code{wgt1}] the final adjusted stage one weight for each site.  The default
is NULL.
\item[\code{x1}] the stage one x-coordinate for location for each site.  The default
is NULL.
\item[\code{y1}] the stage one y-coordinate for location for each site.  The default
is NULL.
\item[\code{popsize}] the known size of the resource - the total number of sampling 
units of a finite resource or the measure of an extensive resource,
which is required for calculation of finite and continuous population 
correction factors for a single-stage sample.  This variable is also 
used to adjust estimators for the known size of a resource.  For a
stratified sample this variable must be a vector containing a value 
for each stratum and must have the names attribute set to identify the
stratum codes.  The default is NULL.
\item[\code{stage1size}] the known size of the stage one sampling units of a two-
stage sample, which is required for calculation of finite and  
continuous population correction factors for a two-stage sample and 
must have the names attribute set to identify the stage one sampling 
unit codes.  For a stratified sample, the names attribute must be set
to identify both stratum codes and stage one sampling unit codes using
a convention where the two codes are separated by the \# symbol, e.g.,
"Stratum 1\#Cluster 1".  The default is NULL.
\item[\code{support}] the support value for each site - the value one (1) for a 
site from a finite resource or the measure of the sampling unit  
associated with a site from an extensive resource, which is required  
for calculation of finite and continuous population correction  
factors.  The default is NULL.
\item[\code{swgt}] the size-weight for each site, which is the stage two size-weight 
for a two-stage sample.  The default is NULL.
\item[\code{swgt1}] the stage one size-weight for each site.  The default is NULL.
\item[\code{unitsize}] the known sum of the size-weights of the resource, which for a 
stratified sample must be a vector containing a value for each stratum 
and must have the names attribute set to identify the stratum codes.  
The default is NULL.
\item[\code{vartype}] the choice of variance estimator, where "Local" = local mean
estimator and "SRS" = SRS estimator.  The default is "Local".
\item[\code{conf}] the confidence level.  The default is 95\%.
\item[\code{cdfval}] the set of values at which the CDF is estimated.  If a set of
values is not provided, then the sorted set of unique values of the
response variable is used.  The default is NULL.
\item[\code{pctval}] the set of values at which percentiles are estimated.  The
default set is: {5, 10, 25, 50, 75, 90, 95}.
\item[\code{check.ind}] a logical value that indicates whether compatability
checking of the input values is conducted, where TRUE = conduct 
compatibility checking and FALSE = do not conduct compatibility 
checking.  The default is TRUE.
\item[\code{warn.ind}] a logical value that indicates whether warning messages were
generated, where TRUE = warning messages were generated and FALSE = warning
messages were not generated.  The default is NULL.
\item[\code{warn.df}] a data frame for storing warning messages.  The default is
NULL.
\item[\code{warn.vec}] a vector that contains names of the population type, the
subpopulation, and an indicator.  The default is NULL.
\end{ldescription}
\end{Arguments}
\begin{Details}\relax
This function calculates an estimate of the deconvoluted cumulative 
distribution function (CDF) for the proportion (expressed as percent) and the 
total of a response  variable, where the response variable may be defined
for either a finite or an extensive resource.  Optionally, for a finite 
resource, the size-weighted CDF can be calculated.  In addition the standard 
error of the estimated CDF and confidence bounds are calculated.  The 
simulation extrapolation deconvolution method (Stefanski and Bay, 1996) is used 
to deconvolute measurement error variance from the response.  The user
can supply the set of values at which the CDF is estimated.  For the 
CDF of a proportion, the Horvitz-Thompson ratio estimator, i.e., the
ratio of two Horvitz-Thompson estimators, is used to calculate the CDF
estimate.  For the CDF of a total, the user can supply the known size of
the resource or the known sum of the size-weights of the resource, as
appropriate.  For the CDF of a total when either the size of the
resource or the sum of the size-weights of the resource is provided, the
classic ratio estimator is used to calculate the CDF estimate, where
that estimator is the product of the known value and the Horvitz-Thompson ratio
estimator.   For the CDF of a total when neither the size of the
resource nor the sum of the size-weights of the resource is provided, the
Horvitz-Thompson estimator is used to calculate the CDF estimate.
Variance estimates for the estimated CDF are calculated using either the
local mean variance estimator or the simple random sampling (SRS) 
variance estimator.  The choice of variance estimator is subject to user 
control. The local mean variance estimator requires the x-coordinate and
the y-coordinate of each site.  The SRS variance estimator uses
the independent random sample approximation to calculate joint inclusion
probabilities.  Confidence bounds are calculated using a Normal
distribution multiplier.  In addition the function uses the estimated 
CDF to calculate percentile estimates.  Estimated confidence bounds for
the percentile estimates are calculated.  The user can supply the set of values
for which percentiles estimates are desired.  Optionally, the user can use the
default set of percentiles.  The function can accommodate a stratified sample.
For a stratified sample, separate estimates and standard errors are calculated
for each stratum, which are used to produce estimates and standard errors for
all strata combined.  Strata that contain a single value are removed.  For a
stratified sample, when either the size of the resource or the sum of the size-
weights of the resource is provided for each stratum, those values are used as 
stratum weights for calculating the estimates and standard  errors for all 
strata combined.  For a stratified sample when neither the size of the resource
nor the sum of the size-weights of the resource is provided for each stratum, 
estimated values are used as stratum weights for calculating the estimates and 
standard errors for all strata combined.   The function can accommodate 
single-stage and two-stage samples for both stratified and unstratified 
sampling designs.  Finite population and continuous population correction 
factors can be utilized in variance estimation.  The function checks for 
compatibility of input values and removes missing values.
\end{Details}
\begin{Section}{Value}
If the function was called by the cont.analysis function, then value is a list
containing the following components:
\Itemize{
\item \code{Results} - a list composed of two objects:
\Itemize{
\item \code{CDF} - a data frame that contains CDF estimates
\item \code{Pct} - a data frame that contains percentile estimates
}
\item \code{warn.ind} - a logical value indicating whether warning messages
were generated
\item \code{warn.df} - a data frame containing warning messages
}
If the function was called directly, then value is a list containing the
following components:
\Itemize{
\item \code{CDF} - a data frame that contains CDF estimates
\item \code{Pct} - a data frame that contains percentile estimates
}
\end{Section}
\begin{Author}\relax
Tom Kincaid \email{Kincaid.Tom@epa.gov}
\end{Author}
\begin{References}\relax
Diaz-Ramos, S., D.L. Stevens, Jr., and A.R. Olsen. (1996).  \emph{EMAP
Statistical Methods Manual.} EPA/620/R-96/XXX.  Corvallis, OR: U.S.
Environmental Protection Agency, Office of Research and Development, National
Health Effects and Environmental Research Laboratory, Western Ecology
Division.
\end{References}
\begin{Examples}
\begin{ExampleCode}
z <- rnorm(100, 10, 1)
wgt <- runif(100, 10, 100)
cdfval <- seq(min(z), max(z), length=20)
cdf.decon(z, wgt, sigma=0.25, var.sigma=0.1, vartype=
  "SRS", cdfval=cdfval)

x <- runif(100)
y <- runif(100)
cdf.decon(z, wgt, sigma=0.25, var.sigma=0.1, x, y, cdfval=
  cdfval)
\end{ExampleCode}
\end{Examples}

