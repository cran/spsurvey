\HeaderA{albersgeod}{Project Albers Projection in Plane to Spheroid}{albersgeod}
\keyword{survey}{albersgeod}
\begin{Description}\relax
Project Albers projection in plane to spheroid models of the globe.
\end{Description}
\begin{Usage}
\begin{verbatim}
albersgeod(x, y, sph="Clarke1866", clon=-96, clat=23,
  sp1=29.5, sp2=45.5)
\end{verbatim}
\end{Usage}
\begin{Arguments}
\begin{ldescription}
\item[\code{x}] Albers x-coordinate vector to be projected to latitude/longitude.
\item[\code{y}] Albers y-coordinate vector to be projected to latitude/longitude.
\item[\code{sph}] spheroid options: Clarke1866, GRS80, WGS84.  The default is 
Clarke1866.
\item[\code{clon}] center longitude (decimal degrees).  The default is -96.
\item[\code{clat}] origin latitude (decimal degrees).  The default is 23.
\item[\code{sp1}] standard parallel 1 (decimal degrees).  The default is 29.5.
\item[\code{sp2}] standard parallel 2 (decimal degrees).  The default is 45.5.
\end{ldescription}
\end{Arguments}
\begin{Details}\relax
Ask Denis White.
\end{Details}
\begin{Value}
A data frame of latitude and longitude projections for x, y Albers coordinates.
\end{Value}
\begin{Author}\relax
Denis White \email{White.Denis@epa.gov}
\end{Author}
\begin{References}\relax
J. Snyder, USGS Professional Paper 1395
\end{References}

