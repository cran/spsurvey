\HeaderA{cdf.test}{Cumulative Distribution Function - Inference}{cdf.test}
\keyword{survey}{cdf.test}
\keyword{distribution}{cdf.test}
\begin{Description}\relax
This function calculates the Wald, Rao-Scott first order corrected (mean 
eigenvalue corrected), and Rao-Scott second order corrected (Satterthwaite
corrected) statistics for categorical data to test for differences between two
cumulative distribution functions (CDFs).  The functions calculates both 
standard versions of those three statistics, which are distributed as 
Chi-squared random variables, plus modified version of the statistics, which 
are distributed as F random variables.
\end{Description}
\begin{Usage}
\begin{verbatim}
cdf.test(sample1, sample2, bounds, vartype="Local")
\end{verbatim}
\end{Usage}
\begin{Arguments}
\begin{ldescription}
\item[\code{sample1}] the sample from the first population in the form of a list
containing the following components:\\
z = the response value for each site\\
wgt = the final adjusted weight (inverse of the sample inclusion probability) for each site\\
x = x-coordinate for location for each site, which may be NULL\\
y = y-coordinate for location for each site, which may be NULL
\item[\code{sample2}] the sample from the second population in the form of a list
containing the following components:\\
z = the response value for each site\\
wgt = the final adjusted weight (inverse of the sample inclusion probability) for each site\\
x = x-coordinate for location for each site, which may be NULL\\
y = y-coordinate for location for each site, which may be NULL
\item[\code{bounds}] upper bounds for calculating the classes for the CDF.
\item[\code{vartype}] the choice of variance estimator, where "Local" = local mean
estimator and "SRS" = SRS estimator.  The default is "Local".
\end{ldescription}
\end{Arguments}
\begin{Details}\relax
The user supplies the set of upper bounds for defining the classes for the
CDFs.  The Horvitz-Thompson ratio estimator, i.e., the ratio of two
Horvitz-Thompson estimators, is used to calculate estimates of the class
proportions for the CDFs.  Variance estimates for the test statistics are
calculated using either the local mean variance estimator or the simple random
sampling (SRS) variance estimator.  The choice of variance estimator is
subject to user control.  The SRS variance estimator uses the
independent random sample approximation to calculate joint inclusion
probabilities.  The function checks for compatability of input values and
removes missing values.
\end{Details}
\begin{Value}
Value is a data frame containing the test statistic, degrees of 
freedom (two values labeled Degrees of Freedom\_1 and Degrees of  Freedom\_2),
and p value for the Wald, mean eigenvalue, and Satterthwaite test procedures,
which includes both Chi-squared distribution and F  distribution versions of 
the procedures.  For the Chi-squared versions of  the test procedures, Degrees
of Freedom\_1 contains the relevant value  and Degrees of Freedom\_2 is set to 
missing (NA).  For the F-based  versions of the test procedures Degrees of 
Freedom\_1 contains the  numerator degrees of freedom and Degrees of 
Freedom\_2 contains the  denominator degrees of freedom.
\end{Value}
\begin{Author}\relax
Tom Kincaid \email{Kincaid.Tom@epa.gov}
\end{Author}
\begin{References}\relax
Kincaid, T.M. (2000). Testing for differences between cumulative distribution
functions from complex environmental sampling surveys.  In \emph{2000
Proceeding of the Section on Statistics and the Environment}, American
Statistical Association, Alexandria, VA.
\end{References}
\begin{Examples}
\begin{ExampleCode}
z <- rnorm(100, 10, 1)
wgt <- runif(100, 10, 100)
sample1 <- list(z=z, wgt=wgt)
sample2 <- list(z=z+2, wgt=wgt)
bounds <- seq(min(sample1$z, sample2$z), max(sample1$z,
   sample2$z), length=4)[-1]
cdf.test(sample1, sample2, bounds, vartype="SRS")

x <- runif(100)
y <- runif(100)
sample1 <- list(z=z, wgt=wgt, x=x, y=y)
sample2 <- list(z=z+rnorm(100), wgt=wgt, x=x, y=y)
bounds <- seq(min(sample1$z, sample2$z), max(sample1$z,
   sample2$z), length=4)[-1]
cdf.test(sample1, sample2, bounds)
\end{ExampleCode}
\end{Examples}

