\HeaderA{sp2shape}{Convert an sp Package Object to an ESRI Shapefile}{sp2shape}
\keyword{IO}{sp2shape}
\begin{Description}\relax
This function creates an ESRI shapefile from an sp package object.  The type 
of shapefile, i.e., point, polyline, or polygon, is determined by the class of
the sp object, which must be either "SpatialPointsDataFrame",
"SpatialLinesDataFrame", or "SpatialPolygonsDataFrame".
\end{Description}
\begin{Usage}
\begin{verbatim}
sp2shape(sp.obj, shpfilename="tempfile", prjfilename=NULL)
\end{verbatim}
\end{Usage}
\begin{Arguments}
\begin{ldescription}
\item[\code{sp.obj}] the sp package object.
\item[\code{shpfilename}] name (without any extension) of the output shapefile.  The
default is "tempfile".
\item[\code{prjfilename}] name (without any extension) of the projection file for the
output shapefile.  The default is NULL.
\end{ldescription}
\end{Arguments}
\begin{Value}
An ESRI shapefile of type point, polyline, or polygon.
\end{Value}
\begin{Author}\relax
Tom Kincaid \email{Kincaid.Tom@epa.gov}
\end{Author}
\begin{References}\relax
ESRI Shapefile Technical Description: 
\url{http://www.esri.com/library/whitepapers/pdfs/shapefile.pdf}
\end{References}
\begin{Examples}
\begin{ExampleCode}
  ## Not run: 
  sp2shape(my.sp.object, "my.shapefile")
  
## End(Not run)
\end{ExampleCode}
\end{Examples}

