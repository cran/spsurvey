\HeaderA{marinus}{Convert Coordinates from Latitude/Longitude to the Equidistant, Cylindric Map Projection}{marinus}
\keyword{misc}{marinus}
\begin{Description}\relax
This function converts x,y coordinates measured in units of latitude and 
longitude, i.e., geographic coordinates measured in decimal degrees, to 
coordinates in the equidistant, cylindric map projection measured in units of
kilometers.  The projection center is defined as the midpoint in latitude-
longitude space.  The map projection is here named after Marinus of Tyre.
\end{Description}
\begin{Usage}
\begin{verbatim}
marinus(lat, lon)
\end{verbatim}
\end{Usage}
\begin{Arguments}
\begin{ldescription}
\item[\code{lat}] vector of latitudes.
\item[\code{lon}] vector of longitudes.
\end{ldescription}
\end{Arguments}
\begin{Value}
A matrix with column names "x" and "y" containing the x and y coordinates 
in the equidistant, cylindric map projection measured in units of kilometers.
\end{Value}
\begin{Author}\relax
Denis White \email{White.Denis@epa.gov}
\end{Author}
\begin{References}\relax
J.P. Snyder. USGS Prof. Paper 1395, p. 90.
\end{References}
\begin{Examples}
\begin{ExampleCode}
lat <- 45 + runif(100, -5, 5)
lon <- 120 + runif(100, -10, 10)
marinus(lat, lon)
\end{ExampleCode}
\end{Examples}

